\chapter*{Jour 5}
Gudule m'a appelé au petit matin, elle m'a laissé un message car je n'arrivais pas à trouver mon téléphone portable à temps. Je l'avais laissé dans la douche, mais je ne me souviens plus trop pourquoi. Un truc à voir avec Archimède, je crois. \\

Gudule s'en va à un festival pour quelques jours, un rassemblement de hippys sur le tard qui veulent s'offrir un mini Woodstock palliatif. Je l'imagine bien les seins nus avec un collier de fleurs et un énorme pétard visé aux coins des lèvres à se déhancher sur de la goa. Je me fais sans doute du mal. \\
Peut-être qu'elle va me ramener un souvenir de son séjour chez les babas, un truc tout bizarre en macramé ou un attrape rêve / poussière indien. \\

Gudule à encore la pêche, c'est dingue comme cette fille déborde de bonne humeur, dommage pour moi que ce ne soit pas contagieux. Elle doit prendre des super céréales comme ils en font la pub à la télé, les super petit déjeuné qui effacent d'un coup les cernes et les gueules de bois. Mes biscuits se cassent toujours quand je les trempe, et j'ai une pâte toute gluante au fond de tasse quand j'arrive au fond. \\

... garder mon chien, le temps que je revienne. \\

Ah. Oui. Je me disais. \\
Je me disais qu'un appel aux aurores ne pouvait pas être innocent. Je déteste les matins, presque autant que les après midi et mes nuits blanches. \\

Gudule est passé en coup de vent, elle s'est penché sur la pointe des pieds pour me déposer un baiser sur la joue et s'en est allé comme elle est venue : tout en sourire et en légèreté. Elle marche pas, elle glisse à quelques centimètres au dessus du sol. C'est plutôt cool. \\

Le bon coté de la chose c'est que je vais voir si un chien pourrait me convenir comme animal de compagnie, et sans passer par la case animalerie. Je ne sais pas combien peut valoir une bête comme ça mais au kilo : ça doit faire son prix. C'est un bas rouge, pur race il paraîtrait, il a un nom mais je ne m'en souvient plus, je suis pourtant sur qu'à un moment ou à un autre Gudule à dut me le souffler. Médor, ça fera l'affaire, appelons le Médor. \\

- Tu viens Médor ?\\
En guise de réponse, l'animal s'est contenté d'esquisser un haussement de babines, m'offrant une vue imprenable sur des canines aussi blanches qu'acérées. Bien à ce que je peut voir, monsieur est susceptible. Heureusement qu'il a un collier avec son nom dessus : Bouba. Une muselière aurait été très assortie mais ce n'est qu'un point de vue très personnel.\\
- Tu viens Bouba ?\\

Je suis allé vite fait chercher des croquettes pour cerbère domestique à la supérette du coin, Bouba commençait à me regarder étrangement et j'ai cru comprendre qu'il avait faim. Quand je suis rentré, il n'avait pas bougé d'un pouce mais il y avait une mare de bave à ses pattes. A intervalle régulier des gouttes tombaient de sa gueule pour aller grossir la flaque. Visiblement la faim guettait l'animal et celui ci me jetait des regards plus que douteux. \\
Je me suis pas fait prier pour remplir sa gamelle, une casserole improvisée en gamelle de fortune pour être exact. La différence n'a semble il pas altéré l'appétit du molosse : il a fait disparaître le contenu de sa gamelle en un tour de langue et il s'est replanté aussitôt sur son arrière train et à replanté dans la foulée son regard affamé dans mes pupilles dilatées. \\
Message reçu, je remplis de suite la casserole. \\
Encore ? Bon elle est peut-être pas si grande que ça, toute compte fait. Et puis on va pas se fâcher pour une bête question de vaisselle. Pas vrai, Bouba ? \\

Pas moins de quatre allés retours entre la cuisine et la gamelle de Bouba ont été nécessaire pour éteindre le regard de l'animal. Une chance pour moi car le sac de croquettes touchait à sa fin. \\

Note pour plus tard, acheter des croquettes pour ours brun. \\
Correction de la note, vérifier s'il existe des croquettes pour grizzly. \\

Finalement, je suis allé me coucher, j'ai laissé Bouba sur place, à sa place. La flemme d'éponger toute la bave à ses pattes, je verrais ça demain matin, avec un peu de chance tout aura séché durant la nuit. \\

Je me suis réveillé en sursaut en sentant une haleine bien chargée sur ma joue. Le genre de cauchemar qui me tire immanquablement du plus profond de mes sommeils. Juste à mon chevet se tenait Bouba tout aussi immobile qu'inquiétant. Ses pupilles rouges plantées en face des miennes. Un court instant j'ai cru que mon cœur s'était mis à battre à l'envers, mon sang effectuant un demi tour au frein à main dans mes veines. J'ai remonté le drap jusqu'au niveau des mes yeux et j'ai retenu mon souffle en attendant la suite. \\
Comme au bout de quelques minutes j'ai pu constater qu'il ne bougeait toujours pas d'un pouce et qu'il n'en prendrait pas le chemin de si tôt, j'ai repris le court de ma respiration. J'ai jamais été bon en apnée et je n'ai jamais aimé le Grand Bleu ni Jean Marc Barre. \\

L'animal ne m'a pas quitté de toute la nuit, il est resté à me veiller à mon chevet, ses yeux constamment braqués sur moi, comme des phares de bagnole sur un lapin. Ce crétin de clébard est insomniaque et moi aussi par rebond. Je vous laisse imaginer que le souffle d'un carnivore n'est pas la plus belle des berceuses que l'on puisse imaginer. \\

Je l'ai finalement déposé dès le lendemain sur le palier de Gudulle, j'ai glissé un mot sous sa porte. \\
Je te ramène ton chien, il me fout les jetons et il me coûte un smic en croquettes. Je crois qu'il faut se résoudre à l'évidence, nous ne sommes pas fait pour vivre ensemble. \\

Personne ne viendra voler un bestiau de cette taille. Même le dernier des suicidaires préférerait se jeter sous un train ou s'ouvrir les poignets à la fourchette à désert plutôt de venir se fourrer entre ses pattes. Le dernier et même l'avant dernier des suicidaires sera mort bien avant de monter les marches jusqu'à l'étage de Gudulle. D'ailleurs je me demande ce qu'il irait faire là bas, il existe des endroits largement mieux pour se foutre en l'air.\\
Rien à craindre du coté de l'enlèvement. Et quand je pense qu'ils y en a qui osent dire que le chien est le meilleur ami de l'homme, franchement j'aimerais pas être leur ami. Non mais je vous jure. \\

Allez tchao et bon vent, Bouba.