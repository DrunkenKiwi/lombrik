\chapter*{Jour 10}
Dur réveil, je ne me souviens plus exactement comment je me suis retrouvé dans mon lit. Un bon point, c'est que je me suis réveillé seul. \\

Un meilleur point s'aurait été de trouver une mannequin hollandaise super sexy et tatouée à mes cotés. \\
Enfin hollandaise, c'est vite dit, suisse ça l'aurait fait aussi. \\

J'ai encore un mal de crane carabiné, j'ai du oublier de fermer la bouteille de dissolvant avant de me coucher. Une négligence qui va me coûter cher. Je sens mon cerveau qui tape sur les bords à chaque fois que je fais un mouvement ou que je me penche un peu vite.\\

Je manque de renverser mes céréales, signe que rien ne va ce matin, et je parviens in extremis à ne pas arroser mon caleçon avec le lait. \\
En levant la tête de mon bol je croise le regard de Findus. Il a réussit à monter sur la table, je lui verse quelques unes de mes céréales à même la table en guise de petit déjeuné. \\

Il y a pire comme voisin de table. \\
Au moins il est silencieux. \\
Et puis il aime les céréales, mes céréales, et c'est plutôt un signe de bon goût. \\

Impossible de mettre la main sur mon téléphone, je ne retrouve pas non plus mon paquet de cigarettes. \\
Tu les aurais pas vu, Findus ? Des fois que t'aurais voulu t'en griller une en douce ou que tu en aurais profité pour appeler ton ancien maître. Remarques je vais pas t'en faire une crise mais si tu peux me les rendre maintenant ce serait cool. \\
Tu m'écoutes Findus ?\\
Il dit rien le bestiau, mais je suis sur qu'avec ses grandes oreilles il n'en a pas perdu une miette. \\

Je croise Findus en sortant de ma douche. \\
Je croise Findus en finissant d'enfiler mon t-shirt. \\
J'ai l'impression qu'il me suit, un peu comme si c'était un petit canard et moi sa mère. Peut-être que les lapins se comportent un peu comme les canards, sauf qu'ils volent pas et qu'ils font moins de bruit, et que je risque moins de l'attraper à déféquer dans ma baignoire. \\

Impossible de mettre la main sur mon téléphone ni mes cigarettes, le besoin de nicotine commence à se faire pressant. Après vingt minutes de recherche, je les retrouve finalement dans le porte savon de la douche. \\
Je me demande comment j'ai bien pu passer à coté et comment je me suis débrouillé pour les poser ici, hier soir. \\
Trop réfléchir me donne la migraine. \\

Je m’apprête à sortir, je ne sais plus pourquoi mais je suis certain qu'une fois dehors cela va me revenir. Et puis l'air du dehors sera certainement bénéfique pour ma céphalée. \\
J'adresse un geste d'au revoir au lapin. A mon lapin, il faudra que je m'y fasse, c'est mon lapin et pas un lapin. Je suis autant son propriétaire que son maître. Encore que je doute que je puisse me faire obéir de Findus. \\

Je me suis arrêté au bout de quelques marches seulement. Le silence devait assourdissant. J'ai fait volte face, j'ai rouvert ma porte pour tomber nez à truffe sur mon colocataire quadrupède. Il n'a rien dit, bien évidement, et je l'ai écouté sans l’interrompre. \\
Le silence du lapin n'a rien à voir avec celui de Gaïl. \\

J'ai finalement cédé à la demande de Findus et je l'ai fourré dans un sac à dos afin qu'il puisse voir par dessus mon épaule. Je n'aime pas que l'on regarde par dessus mon épaule mais je suis prêt à faire une exception pour un lapin. Il y a peu de chances qu'il sache lire après tout. C'est un lapin, devrais-je vous le rappeler. \\

J'ai guetté un peu pour éviter ma voisine. Elle ne m'adresse plus la parole depuis que Snow s'est offert un repas gastronomique sur son chat mais peut-être que si elle me voit avec une paire de grandes oreilles qui dépasse de mon sac elle risquerait de vouloir engager une conversation aussi insipide qu'ennuyeuse. J'ai pas envie de mourir d'ennui en écoutant une petite vieille radoter sur les animaux de compagnie. \\
Ci-gît Néon, mort la gueule grande ouverte comme un poisson rouge sur les marches de son immeuble.\\
Je ne vois, ni ne sent, de petit vieux à proximité mais par acquis de conscience : je descends les marches sur la pointe des pieds. \\

J'ai traversé la rue, la tête plantée dans mes pensées, à essayer de me souvenir ce que je devais faire aujourd'hui. \\
Un truc important : c'est sur et je sais que je vais me faire sonner les cloches si j'oublie de quoi il en retourne. De toute façon quand j'oublie un truc c'est toujours un truc important, jamais une broutille, non juste un truc vital. \\

J'étais tellement absorbé par l’effeuillage de mon agenda intérieur que je n'ai pas entendu les deux premiers appels. Généralement quand une personne crie votre nom dans la rue c'est pour attirer votre attention ou alors c'est qu'elle souffre d'un sérieux problème et dans ce cas il faut mieux continuer son chemin comme si de rien n'était. \\

\begin{quote}
- Néon tu pourrais m'attendre quand même ? \\
\end{quote}

Néon : c'est mon nom, enfin mon surnom, c'est uniquement par ce pseudonyme que je me reconnais. Mon prénom ne sert qu'à mon père et à l'état civil. Et encore. \\
Néon : c'est moi, sauf si la voix s'est mise à parler aux lampadaires mais ce serait dramatique. Surtout en plein jour, la nuit avec trois grammes dans les veines, je dis pas. \\
Enfin si je trouverais ça con mais j'irais pas jusqu'à le dire à voix haute. \\

La propriétaire de la voix s'est littéralement plantée face à moi, de sorte qu'il était rigoureusement impossible de l'éviter. N'ayant pas l'habitude de piétiner les gens, j'ai donc pilé net en face d'elle. \\
Elle.\\
Je faisait face à une bonnet noir orné d'un semblant de tête de mort issue d'un drapeau pirate d'où émergeait des mèches de cheveux roux et un énorme sourire qui fendait un visage aux pommettes poivrées de tâches de rousseurs. \\

\begin{quote}
- Salut Néon, salut Findus, franchement tu aurais pu m'attendre un peu. Tu m'avais pas entendu, pourtant j'ai crié fort. Tu as l'air bien pressé, ça te dérange pas si je t'accompagne un peu ?\\
- euh ...
\end{quote}

C'est toujours pareil, à chaque fois que je suis sensé dire un truc un temps soit peu spirituel, j'articule une voyelle solitaire et j'en reste là. \\
Les conversations c'est pas trop mon truc, surtout quand les gens parlent tout le temps. Ils commencent une phrase et j'ai même pas le temps de répondre qu'ils ont déjà enchaîné sur une nouvelle sentence : du coup je sais même plus quoi dire et je passe pour un imbécile. \\
Elle connaît mon nom, donc elle doit me connaître et sur un principe de réciprocité bien éprouvé : je devrais être capable d'en dire autant. Sauf que là, je sèche. \\

\begin{quote}
- Excusez moi mademoiselle, c'est pas que je veuille vous draguez, enfin sauf si vous êtes partante mais là faudrait qu'on en discute avant, mais c'est quoi déjà votre nom ?
\end{quote}

Elle s'est fendue d'un large sourire et m'a tendue sa main. \\
J'ai regardé sa main un court instant avant de comprendre qu'il fallait que je l'attrape pour la secouer vigoureusement. \\

\begin{quote}
- Je suis La Mort, on s'est croisé quand tu es passé chez Jérémie pour prendre son lapin. \\
- Ah, oui bien sur ... \\
- Il m'a demandé de dire à Findus qu'il ne reviendrait pas de l’hôpital et qu'il fallait que tu t'en occupe maintenant. \\
\end{quote}

La Mort, rien que ça. Et moi qui ai cru un court instant tomber sur une déséquilibrée me voilà grandement rassuré. Je suis en train de faire la causette avec la grande faucheuse au sujet du lapin dans mon sac à dos. \\
C'est cool, tout va bien. Je respire un grand coup et je fais le vide à l'intérieur. \\

\begin{quote}
- Vous êtes la mort comme ça. \\
- Non mais t’inquiètes pas je viens pas pour toi, enfin si mais pas pour venir te chercher, enfin pas encore tu comprends. \\
- Dans les grandes lignes, seulement ... c'est juste que c'est pas exactement comme ça que je t'imaginais. \\
- Tu pensais que je ressemblais plus à un grand squelette en capuche avec un faux dans la main ? Faut vivre avec ton temps Néon. Et franchement tu trouves pas que je suis mieux comme ça ? \\
\end{quote}

Sur ces mots elle effectue une volte afin que je puisse l'admirer sous toutes ses coutures. Il faut reconnaître qu'elle n'est pas complètement dépourvue de charmes, avec ses petites fesses rebondies et ses tâches de rousseurs. \\
Faut te reprendre Néon c'est sur la mort que tu phantasmes. \\

\begin{quote}
- Tu veux voir mes seins ? Dit elle en prenant en mettant en coupe ses mains sur sa poitrine. \\
- Non, c'est très sympa mais ça ira ... \\
- T'as tord pourtant, c'est exactement les mêmes que ceux d'Hélène de Troie. \\
- Tu sais je suis pas trop calé en matière de présentatrices télé ... \\
- Hélène de Troie, dans l'Iliade. Voici donc les seins pour lesquels se sont armés tant de navire. \\
- Il a vraiment écrit ça Homère ? \\
- Écrit peut-être pas mais il l'a dit j'en suis sure j'étais là. Remarque il avait un peu trop bu alors peut-être qu'ils ont coupé ça au montage ensuite. Enfin c'est pas grave, si un jour tu changes d'avis : ils sont toujours là. 
\end{quote}

Joignant une nouvelle fois le geste à la parole elle pointa ses deux index en direction de ses tétons. \\
Si ses seins sont bien ceux d'Hélène je me demande à qui appartiennent donc ses fesses. Je suis certain, et Gaïl me le confirmera sans doute si je lui demandais, que ce n'est pas le genre de question qu'il convient de poser à une damoiselle. \\
La mort est femelle, j'en ai maintenant l'absolue certitude, et je dois ajouter qu'elle à un beau cul ce qui est en revanche une nouveauté. \\

Nous avons conversé un long moment, ses fessiers sont ceux d'une esclave romaine morte des suites d'une mauvaise grippe. Les fossettes sacro-iliaques sont celles d'une actrice hollywoodienne des années cinquante mortes des suites d'une mauvaise chute de limousine. Un escarpin qui se casse sur un trottoir et la nuque qui se brise dans la foulée et l'encadrement de la portière. \\

Je crois qu'elle est cool, enfin pour la mort je veux dire. 