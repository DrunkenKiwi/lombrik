\chapter*{Jour 10}
Dur réveil, je ne me souviens plus exactement comment je me suis retrouvé dans mon lit. Un bon point, c'est que je me suis réveillé seul. \\

Un meilleur point s'aurait été de trouver une mannequin hollandaise super sexy et tatouée à mes cotés. \\
Enfin hollandaise, c'est vite dit, suisse ça l'aurait fait aussi. \\

J'ai encore un mal de crane carabiné, j'ai du oublier de fermer la bouteille de dissolvant avant de me coucher. Une négligence qui va me coûter cher. Je sens mon cerveau qui tape sur les bords à chaque fois que je fais un mouvement ou que je me penche un peu vite.\\

Je manque de renverser mes céréales, signe que rien ne va ce matin, et je parviens in extremis à ne pas arroser mon caleçon avec le lait. \\
En levant la tête de mon bol je croise le regard de Findus. Il a réussit à monter sur la table, je lui verse quelques unes de mes céréales à même la table en guise de petit déjeuné. \\

Il y a pire comme voisin de table. \\
Au moins il est silencieux. \\
Et puis il aime les céréales, mes céréales, et c'est plutôt un signe de bon goût. \\

Impossible de mettre la main sur mon téléphone, je ne retrouve pas non plus mon paquet de cigarettes. \\
Tu les aurais pas vu, Findus ? Des fois que t'aurais voulu t'en griller une en douce ou que tu en aurais profité pour appeler ton ancien maître. Remarques je vais pas t'en faire une crise mais si tu peux me les rendre maintenant ce serait cool. \\
Tu m'écoutes Findus ?\\
Il dit rien le bestiau, mais je suis sur qu'avec ses grandes oreilles il n'en a pas perdu une miette. \\

Je croise Findus en sortant de ma douche. \\
Je croise Findus en finissant d'enfiler mon t-shirt. \\
J'ai l'impression qu'il me suit, un peu comme si c'était un petit canard et moi sa mère. Peut-être que les lapins se comportent un peu comme les canards, sauf qu'ils volent pas et qu'ils font moins de bruit, et que je risque moins de l'attraper à déféquer dans ma baignoire. \\

Impossible de mettre la main sur mon téléphone ni mes cigarettes, le besoin de nicotine commence à se faire pressant. Après vingt minutes de recherche, je les retrouve finalement dans le porte savon de la douche. \\
Je me demande comment j'ai bien pu passer à coté et comment je me suis débrouillé pour les poser ici, hier soir. \\
Trop réfléchir me donne la migraine. \\

Je m’apprête à sortir, je ne sais plus pourquoi mais je suis certain qu'une fois dehors cela va me revenir. Et puis l'air du dehors sera certainement bénéfique pour ma céphalée. \\
J'adresse un geste d'au revoir au lapin. A mon lapin, il faudra que je m'y fasse, c'est mon lapin et pas un lapin. Je suis autant son propriétaire que son maître. Encore que je doute que je puisse me faire obéir de Findus. \\

Je me suis arrêté au bout de quelques marches seulement. Le silence devait assourdissant. J'ai fait volte face, j'ai rouvert ma porte pour tomber nez à truffe sur mon colocataire quadrupède. Il n'a rien dit, bien évidement, et je l'ai écouté sans l’interrompre. \\
Le silence du lapin n'a rien à voir avec celui de Gaïl. \\

J'ai finalement cédé à la demande de Findus et je l'ai fourré dans un sac à dos afin qu'il puisse voir par dessus mon épaule. Je n'aime pas que l'on regarde par dessus mon épaule mais je suis prêt à faire une exception pour un lapin. Il y a peu de chances qu'il sache lire après tout. C'est un lapin, devrais-je vous le rappeler. \\

J'ai guetté un peu pour éviter ma voisine. Elle ne m'adresse plus la parole depuis que Snow s'est offert un repas gastronomique sur son chat mais peut-être que si elle me voit avec une paire de grandes oreilles qui dépasse de mon sac elle risquerait de vouloir engager une conversation aussi insipide qu'ennuyeuse. J'ai pas envie de mourir d'ennui en écoutant une petite vieille radoter sur les animaux de compagnie. \\
Ci-git Neon, mort la gueule grande ouverte sur les marches de son immeuble.\\
Je ne vois, ni ne sent, de petit vieux à proximité mais par acquis de conscience : je descends les marches sur la pointe des pieds.