\part{Préambule}

Afin de dissiper dès maintenant toute ambiguïté, sachez que je ne suis pas l'auteur de ce livre. Si toutefois on peut appeler ça un livre. Je ne suis pas littéraire, ni de formation ni de cœur, je suis certain qu'il existe un terme plus juste pour désigner la chose que vous tenez dans les mains. Ce n'est sans doute pas un roman ni même un essai ou un pamphlet, c'est tout juste un ensemble de feuilles reliées les unes aux autres. Livre m'a semblé être le mot le plus juste, si toutefois vous en connaissez de meilleur je vous laisse effectuer de vous même la correction. \\

Ce qui va suivre est la retranscription de ce qui pourrait s'apparenter à un journal intime. Je dois vous avouer que je suis resté parfois un peu perplexe sur la véracité de ce qui est relaté. Il semblerait que l'auteur souffre de quelques troubles mentaux, une inadaptation complète à la vie en société, une forme très particulière de synesthésie ou un sens des comparaisons particulièrement douteux.\\

Je retranscris le texte tel qu'il est, sans faire la moindre césure ni la moindre censure d'aucune sorte. Je dois pour autant me défendre par avance et préciser que ce ne sont ni mes propos ni mes opinions et qu'ils n'engagent que leur auteur. Je me suis permis toutefois d'ajouter des notes de fin de page là où il me semblait qu'un peu d'explication supplémentaires était nécessaire. L'auteur utilise occasionnellement des mots empruntés au japonais et il lui arrive parfois même de pousser le vice jusqu'à utiliser des idéogrammes. Je n'ai pas modifié son texte et je me suis simplement contenté de les traduire au plus juste.\\

Le plus simple serait sans doute de commencer par le commencement. J'ai reçu un beau matin un imposant carton par la poste, ce qui m'a par ailleurs beaucoup surpris car je n'avais rien commandé par correspondance. En ouvrant le carton j'ai put constater qu'il contenait une série de cahiers d'écolier soigneusement alignés au fond du carton, un bocal bien protégé avec du papier bulle contenant ce qui ressemblerait du sable rouge vif. Une enveloppe en papier kraft était scotchée sur le dessus, sans doute à mon attention. Je me suis empressé de l'ouvrir pour lire son contenu d'une main fébrile. Je vous en livre le contenu.\\

\begin{quote}
Salut, \\

je dois partir vite, plus ou moins loin et plutôt longtemps et je ne peux pas m'encombrer de tonnes de bagages. Je vais en refiler une partie aux bonnes œuvres, dès que j'en aurais trouvé quelques unes. J'ai donnée la meilleure partie à Gaïl et à Gudulle, je leur dois bien ça. \\

Il me reste ça, et comme je savais pas à qui les refiler : et bien c'est tombé sur toi. Je crois pas te connaître mais peut-être qu'on s'est déjà croisé par hasard. C'est pas complètement impossible, j'oublie tout le temps les visages et les noms. Si c'est le cas, j'espère que tu m'en voudras pas trop. \\
Bon tout ou presque est là, je pense que tu sauras quoi en faire. Au pire demande à Gaïl. Elle est toujours de bon conseil, tu verras. D'ailleurs c'est elle qui m'a conseillé de faire ça, je crois même qu'elle m'a expliqué pourquoi. Je t'assure que sur le coup j'ai trouvé son idée sacrément bonne mais je ne me souviens plus des détails. \\
Enfin si ça se trouve t'es sans doute aussi intelligent qu'elle et tu comprendras tout seul sans qu'on ai besoin de t'expliquer. \\

Bon, sur ce c'est pas que je m'ennuie mais je peux pas rester trop longtemps, le temps presse et il faut que j'y aille. C'est pas qu'on m'attend quelque part mais c'est tout comme.\\

Je te laisse la clef comme convenu, n'oublie pas de refermer derrière toi, pour les courants d'air.\\

Salut et encore merci pour le coup de main.\\
Neon.\\
\end{quote}

Je suis resté un long moment, à lire et à relire la lettre sous toutes ses coutures pour essayer de comprendre qui en était l'auteur et ce que j'étais sensé faire de ce foutu colis. J'ai vérifié l'adresse sur le colis et vraisemblablement il n'y avait aucune erreur de livraison : c'était bien mon nom et mon adresse. \\

J'ai parcouru la lettre une bonne dizaine de fois, j'ai épluché en détails tout les manuscrits, j'ai fouillé les moindres recoins du colis sans pour autant réussir à trouver la réponse à ma principale question.\\

Pourquoi moi ? \\

Comme je vous l'ai dit au préalable: je ne suis ni critique ni éditeur. Les seuls livres que j'ai encore le temps de parcourir sont des manuels informatiques aussi soporifiques qu'un solide coup de matraque derrière les oreilles. Je me contente juste de mettre bout à bout des petits morceaux de code pour en faire des gros programmes et de refiler le tout à des gros clients pleins aux as qui rêvent de s'enrichir encore un peu plus. \\
Ça ne vous semble peut-être pas très moral comme travail, mais je n'ai que faire de votre avis. J'ai troqué ma conscience contre un crédit sur vingt ans et une place à peu près confortable dans la grande société de consommation. La conscience est un luxe que je me permettrais peut-être sur mes vieux jours, s'il m'est donné d'en avoir. Le monde n'est pas à la hauteur de mes attentes : si cela peut vous rassurer sur mon niveau de conscience. \\
Mes idéaux de jeunesses ont soigneusement été remisés dans un placard pour que je puisse éventuellement les parcourir pour meubler mes vieux jours.\\

Vous conviendrez comme moi que les intentions de Neon concernant ses manuscrits sont pour le moins obscures. Je me demande s'il souhaitait que je les conserve précieusement au secret en attendant son retour ou s'il souhaitait au contraire que je les rendent publiques. \\

J'ai remisé le bocal sur une étagère, hors de porté de mon turbulent fiston. Je ne suis pas certain du contenu du bocal, on dirait du sable ou des pigments mais n'étant sur de rien je préfère le tenir à distance.\\

J'ai scanné toutes les pages des cahiers, les unes après les autres, un véritable travail de titan. J'ai retouché les pages abîmées pour retirer les taches de café et les traces de peintures. J'ai retiré numériquement les tickets de caisse collés au milieu des pages, les crobards à la va vite, les gribouillis en encarts et les ratures. \\

J'ai mouliné le tout dans un logiciel de reconnaissance de caractères de mon cru pour en extraire le texte. Une chance pour moi que Neon écrive en lettres d'imprimerie sinon la tache aurait été impossible. \\

J'ai relu et remis en page et recompilé le tout pour en faire un ensemble un tant soit plus cohérent, puis je l'ai expédié sur quelques sites d'édition en ligne. \\

C'est tout en ce qui me concerne, en vous souhaitant une bonne lecture.