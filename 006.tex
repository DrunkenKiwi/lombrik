\chapter*{Jour 6}
J'ai toujours pas trouvé mon animal de compagnie. \\

Je désespère pas pour autant, faut absolument que j'avance : je dois voir mon psy demain. Je suis certain qu'il serait soulagé d'apprendre que j'ai réussi à caser une petite bestiole dans ma petite vie. Je suis sur qu'il sourirait de toutes ses dents gâtées et qu'il poserait sa grosse main gluante sur mon épaule pour m'assener un truc dans le genre du : \\
Félicitations Nicodème, je suis fier de voir que vous franchissez une à une les marches qui vous mènent vers la sociabilisation.\\

J'aime pas qu'on me touche et j'aime pas non plus que l'on m'appelle par mon prénom. Je pensais qu'un bon psy aurait compris depuis longtemps cette évidence. Suivez le cheminement de ma pensé si je vais pas trop vite pour vous. Sinon je suis au regret de vous annoncer que vous êtes un attardé moyen mais gardez espoir vous êtes très nombreux dans ce cas, alors vous allez pouvoir trouver facilement un copain.\\

Y a que mon père pour m'appeler Nicodème et encore c'est juste parce-qu'il à choisit lui même mon prénom et qu'il ne veut toujours pas admettre qu'il s'est planté. Il aurait pu choisir un prénom commun, un truc qui passe partout : et bien non monsieur mon père à voulu faire une fois de plus le mariole pour se faire remarquer. Les meilleures blagues ne durent pas trois décennies, à bon entendeur Papa.\\

En fait, je ne me soucie pas trop de ce que mon réducteur de cervelle va trouver à me dire. Il passe son temps à faire semblant de m'écouter mais je reste persuadé qu'il pense à autre chose et qu'il attend simplement que mon heure d'analyse s'écoule pour pouvoir encaisser son chèque. \\

Je me demande ce qu'en pense Gaïl ? \\
Gaïl ne répond pas, et son répondeur ne m'est d'aucune aide. \\
Son répondeur me ressasse toujours la même litanie, elle n'est pas disponible pour l'instant mais laisser un message avec vos coordonnées et elle vous rappellera au besoin. C'est une voix enregistré, c'est même pas celle de Gaïl. C'est nul. \\

Finalement je me suis décidé à retourner dans une animalerie. Pas la première, je crois qu'ils ne veulent plus me voir depuis que je leur ai rendu Snow. Je ne saurais vous dire pourquoi mais je suis persuadé que ce serait une très mauvaise idée que de remettre les pieds là bas.\\

Heureusement qu'il y a d'autres animaleries dans cette ville. On trouve plein de choses dans les grandes villes, et parfois c'est cool. Parfois non. Ça dépends principalement de ce que l'on cherche. \\

Je me suis donc rendu à la deuxième animalerie de l'annuaire. Vous vous êtes jamais dit que les pages jaunes devrait classer leurs adresses de manière un tant soit peu plus intelligente, genre par popularité plutôt que par ordre alphabétique ? Le pauvre type qu'a une enseigne qui commence par un Z doit pas voir passer grand monde. Alors que si ça trouve son boui-boui n'est pas plus infâme que la tête de liste. Dans mon cas, le cas présent, ça n'a pas beaucoup d'importance, toutes les animaleries doivent bien se valoir. \\

Je suis entré, et j'ai attendu. \\
J'ai attendu un peu ... \\
puis longtemps ... \\
j'étais sur le point de faire demi tour quand la vendeuse est sortie de nulle part pour venir à ma rencontre. La cinquantaine mal encaissée, le regard sec d'une fouine et la démarché pincée d'une aristocrate anglaise, je doute fort que la patience soit son fort. Et je ne saurais pas plus étonné de savoir que je fais déjà figure d'importun à ses yeux, malheureusement pour elle je risque fort de lui donner raison dans quelques instants. \\

\begin{quote}
- Vous cherchez quelque chose de particulier, monsieur ? \\
- Un animal de compagnie, mais je me suis pas encore arrêté sur un modèle bien précis. Un truc pas trop encombrant et pas trop menaçant si vous avez. \\
\end{quote}

La dite vendeuse est resté figée sur place, en ignorant totalement ma réponse. C'était déplacée à ce point comme réflexion ? Peut-être s'attendait elle à ce que je bredouille un truc et que je parte sans demander mon reste ? \\
Je crois pas, elle est vendeuse. Son boulot c'est bien de me vendre des trucs, en l'occurrence des animaux, pas de me pousser vers la porte. \\
Peut-être ai-je omis une tournure de politesse dans ma phrase ?\\
A part le bonjour, je vois pas trop. Mais d'un autre coté : elle m'a pas saluée non plus. Donc au pire c'est elle qui a commencé et pas moi. \\

En fin de compte elle n'aurait pas du m'arrêter dans ma volte et me couper dans mon élan. Je lui ai tourné le dos et je suis reparti, direction la sortie sans escale ni ravitaillement. \\

Gaïl ne répond toujours pas sur son portable. J'aime de moins en moins son répondeur. \\

Si ça se trouve ce n'est pas une vraie vendeuse mais juste une mécanique, un automate, un canard de Vaucanson amélioré. Elle a peut-être un trou dans le dos pour y mettre un clef pour la remonter, un peu comme les vieux réveils matin.\\

Je ne sais pas trop comment j'en suis venu à penser à un piaf comme animal de compagnie mais tout compte fait je me suis rendu au supermarché du coin. Je me suis dirigé au fond, au rayon "viande" sous section "volaille" pour y faire mon choix. \\
Je suis resté sans doute un long moment à regarder les volatiles sagement alignés dans leurs barquettes respectives. Ça fait tout drôle de regarder ces oiseaux tout nu, sans plumes, ni tête d'ailleurs, bien serrés sous l'opercule plastique. \\

Il y en a de toutes les tailles : de l'imposante dinde à la minuscule caille. Certaines d'entre elles sont estampillées d'un macaron tricolore sur l'emballage, d'autres d'une bague à la patte. On se croirais dans une morgue pour volatiles plus que dans un étalage de boucherie. \\
A un certain degré c'est beau, pourtant je suis sur que celui ou celle qui fait la mise en rayon au petit matin est à des lieues de s'imaginer qu'il ou elle contribue à la création d'une œuvre d'art. J'ai pris une photo avec mon téléphone portable, pour pouvoir en faire une toile plus tard. \\

J'ai pas su résister, faut dire qu'ils sont fort les types du marketing des grandes surfaces. Je sais pas exactement comment ils s'y prennent, s'ils vaporisent des phéromones dans la climatisation ou s'il glissent des messages subliminaux dans la musique d'ambiance. Peut-être les deux à la fois. Tout est il que je ne suis pas ressorti de là les mains vides.\\

Gaïl, c'est encore moi. Toi qu'est douée à tout plein de choses, tu sauras pas cuisinier une volaille par hasard ?