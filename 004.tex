\chapter*{Jour 4}
Snow manque à l'appel. \\

J'ai essayé vainement de l’appeler par son petit nom, j'ai même poussé le vice jusqu'à agiter une saucisse de Strasbourg un peu partout dans mon appartement pour essayer de le faire venir mais en vain. \\
Il m'a pas laissé de mot, mais je m'en doutais un peu quand même. Les furets sont-ils des grands habitués du coup des croissants ou est ce l'apanage exclusif des grands singes ? \\

J'ai jeté un coup d’œil par la fenêtre, juste histoire de vérifier un drôle de pressentiment, une vague histoire de déjà vu à quatre pattes. J'ai pas vu de masse suspecte sur la chaussé donc je suis retourné à mon petit déjeuné. \\
Le rituel des céréales du matin ne souffre d'aucune interruption, fusse la fin du monde. Cette certitude n'est visiblement pas acquise de ma voisine d'en dessous, le bon sens devrait être un peu mieux réparti dans les étages, tout le monde y gagnerait certainement. La dite voisine n'avait rien trouvé de mieux à faire que de hurler à la mort pour une raison qui m'est tout à fait inconnue et qui n'aurait jamais réussie à éveiller ma curiosité. La curiosité tue le chat aussi sûrement qu'une chute du neuvième étage, j'en sais quelque chose croyez moi sur parole. \\

Voilà que l'on tambourine à ma porte maintenant, à croire que le monde entier s'est ligué pour ne pas me laisser déjeuner en paix. \\

Les coups ne voulant pas s'atténuer d'eux même, on est souvent surpris de l'endurance des nuisibles domestiques, j'ai remisé mon bol dans un coin de la table et je suis allé voir de quoi il en retournait exactement. \\
Ma voisine du dessous l'auriez vous cru ? \\
Toute retournée, l'air encore plus incongrue qu'à son habitude avec ses bigoudis visées sur les maigres cheveux blancs qui parsèment encore son crane. Sa mine de petite vieille tellement déconfite que l'on aurait cru un modèle de Picasso (période cubiste, bien sur) sorti de sa toile pour venir me hanter de bon matin et mauvais pied. \\
La voilà maintenant qui s'agrippe à mon col et qui approche dangereusement son visage du mien, on se croirait dans un film de Romero.\\

- Le monstre : il a tué mon Charly ... c'est horrible. \\

Je confirme je suis bien dans un film de Romero. A choisir j'aurais préféré un film muet, contemplatif, avec le strict minima d'action et une foison de plans fixes en noir et blanc. Les films de zombies c'est trop hipster pour moi, je passe mon tour.\\

- Mon Charly ! Il l'a tué le monstre ... \\

C'est dommage, c'est sur.\\
Enfin surtout pour Charly, je suppose. Surtout s'il est mort dans l'affaire. \\
C'est pas comme si j'avais un paquet de choses terriblement urgentes à faire mais présentement j'ai pas trop vocation à être une éponge pour petites vieilles hystériques en larmes. Alors si vous voulez bien vous poussez un peu je vais vous montrer l'extérieur de ma porte.\\
Rien à faire ... oui je sais qu'il est mort, et si vous voulez mon avis : il va le rester longtemps. \\

Allez savoir comment, je me suis retrouvé sans trop comprendre un étage plus bas : dans une pièce encombrée de babioles roses et poussiéreuses : le lieux du crime. Au centre de la pièce dans une marre de sang rouge noirâtre gisait le cadavre d'un chat obèse, passablement éventré avec l'arrière train d'un furet albinos qui dépassait des entrailles du félin.\\
Je crois que la cause du décès est aussi évidente que le coupable. Même avec un bon avocat, la cause va être ardue à défendre. \\
Le petit animal dormait toujours du sommeil du repus, la tête dans ce qui fut un chat, visiblement complètement insensible au drame qui se jouait non loin de lui. J'aurais juré l'entendre ronfler.\\

La vieille mère Michelle hurlait à la mort de son chat.\\

Snow roupillait. \\

Je suis sorti, et j'ai fait ce que je fais toujours dans les situations critiques où je ne sais pas quoi faire. J'ai appelé Gaïl, pour lui faire un rapide point sur la situation et m'en remettre à ses bons conseils.\\
Gaïl est toujours de bon conseil, mais je crois que je vous l'ai déjà dit. Mais peut-être que vous ne vous en souveniez plus ? J'ai donc écouté avec la plus grande attention les conseils qu'elle m'a prodigués. Avec la petite vieille qui hurlait à s'en rompre les cordes vocales en musique de fond ça n'a pas été une mince partie de plaisir, je vous prie de me croire.\\

\begin{itemize}
  \item Nier, c'est très important. C'est même le plus important. Nier tout, en bloc et sans la moindre hésitation, du plus important au détail le plus insignifiant : nier. \\
  Ce n'est pas mon furet, je n'ai jamais eu de furet et je ne sais même pas à quoi cela ressemble. Et quand bien même j'aurais eu un furet vous vous doutez bien que je ne l'aurais jamais, au grand jamais, laissé sortir de mon appartement.
  \item Compatir avec la victime. \\
  La compassion c'est pas mon fort mais je vais essayer de faire au mieux ou plus exactement de parer au moins pire. Je suis désolé pour votre chat (Gaïl m'a formellement interdit de faire la moindre allusion au parquet, même s'il va être dur à ravoir avec une tâche pareille).\\
  \item Se poser en solution, ça évite aux gens de penser que vous êtes plus ou moins affilié au problème, surtout si c'est effectivement le cas. \\
  M'occuper du cadavre c'est dans mes cordes, j'ai des sacs poubelles sous l'évier de la cuisine et j'ai une super serpillière toute neuve. Faut dire aussi que je suis plus à l'aise avec les morts qu'avec les vivants. Je fais parfois des gaffes avec les seconds sur les premiers mais jamais l'inverse. ... c'est peut-être pas très clair comme phrase.\\
  Mais je suis sur que vous m'avez compris et c'est ça le plus important. \\
\end{itemize}

J'ai bazardé Charly dans la benne des ordures ménagères de l'immeuble, je suis pas doué en tri sélectif des déchets. Les chats ne se recycle plus, surtout depuis que les cordes de violons ne se font plus en intestins de félin. De toute façon des intestins il lui en restait plus beaucoup au Charly. \\
J'ai mis une petit croix en bois sous le tilleul du square, il n'est pas exactement enterré là mais je pense que ce n'est qu'un ridicule détail. L'arbre en question est dans la cours intérieur et ma fenêtre donne sur le coté rue, je n'ai donc aucune chance d'avoir la petite vieille veuve d'un chat en larmes sous mes carreaux. \\

J'ai ramené Snow au magasin. Il avait la fourrure rouge de la truffe aux oreilles et un semblant de sourire mesquin sur les babines. \\

Le type du magasin n'a pas voulu me le reprendre, il m'a dit qu'ils ne reprenaient pas les furets. Les animaux du magasin sont tous certifiés neuf sans exception. La caisse enregistreuse à visage humain n'a rien voulu savoir. \\

Pas cool, le gars. \\

J'ai laissé Snow dans la cage des lapins et je suis vite parti sans me retourner. Au revoir et amuse toi bien Snow. \\

Il n'y a pas de petit plaisir dans l'existence.