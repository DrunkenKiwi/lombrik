\chapter*{Jour 1}
Bonjour journal.\\

Je ne sais pas trop quoi dire, c'est la toute première fois de mon existence que j'ai un journal intime. A vrai dire c'est même pas mon idée, c'est Gudulle qui à tenu à me l'offrir en me disant que cela m'aiderait sans doute à canaliser mes angoisses. Elle est sympa dans le fond Gudulle, et pas que dans le fond d'ailleurs. Même si elle vise bien souvent à coté de la plaque : il paraîtrait que c'est pour la bonne cause alors il faut lui pardonner. Je suis sur qu'il si elle arrêtait de se coucher avec des hordes de connards : elle aurait fait une super bonne sœur. Et j'aurais sans doute tenté ma chance avec elle aussi. \\
Donc Gudulle m'a offert un super cahier sans spirale de chez Franprix avec une jolie couverture vert olive et le super crayon plume en acier chromé qui va avec. Super. Ça tombe bien j'aime pas les cahiers à spirale : quand j'étais petit j'en avais un et les pages s'arrachaient toujours quand on écrivait dessus. C'est vraiment pas pratique, je suis sur que l'inventeur du cahier à spirale avait autre chose en tête quand il a déposé son brevet. Il devait penser à son repas du midi ou il rêvassait en regardant sa voisine. S'il s'était un peu plus appliqué sur son prototype : il aurait sans doute fait un truc solide et pratique qui aurait servi à des générations entières jusqu’à l'arrivée de la tablette numérique. Au lieu de ça, il nous a fourgué un truc juste à mi chemin entre l'inutilisable et le sadisme. Et dire que j'avais des professeurs de sciences physiques qui m'imposaient ce genre de cahier. Il n'y a qu'un prof de sciences physiques pour avoir des idées pareilles.\\
Le plume est bien aussi, il a pas l'air de baver et j'arrive même à relire mes gribouilles après. Elle à même pensé à me mettre un petit sac en plastique avec des cartouches d'encre d'avance pour être sur que je ne tomberais pas en panne sèche sous peu. Elle pense à tout Gudulle. \\

Ça fait bizarre de parler comme ça en pensant qu'il y aura sans doute jamais personne pour me lire. C'est un peu comme parler tout seul le soir dans son lit, sauf que là ça fait pas de bruit et qu'on vous prend moins pour un demeuré. Tu vois lecteur imaginaire, je te fais un coucou de la main et tu l'aurais jamais su si je te l'avais pas dit. Enfin écrit, pas dit. Mais on se comprend, j'espère. Je suis sur que tu as hoché la tête, donc je vais croire que oui.\\

J'ai parlé de l'idée de Gudulle à mon psy, qui est un con et qui à trouvé que l'idée était bonne. L'un dans l'autre je ne sais pas si cela doit me rassurer ou pas. Il dit que le fait de coucher mes pensées par écrit pourrait faire une excellente catharsis. C'est pas pour autant qu'il va me faire un rabais sur les prochaines séances. Ça doit pas être dans la déontologie des psy, en gros on peut essayer de se soigner tout seul mais c'est pas une raison suffisante pour se dispenser de les voir et surtout de les payer.\\
Gudulle est pas psy, ni bonne sœur mais elle pourrait si elle voulait, et elle me l'a donné le cahier sans rien me demander en retour. J'angoisse toujours un peu avec les cadeaux, car les gens ne font jamais vraiment rien gratuitement, par simple altruisme. Il y a toujours une pensée en dessous, l'attente d'un retour d'ascenseur, d'un service en demeure, d'une sorte d'épée de Damocles en suspens qui menacerait de tomber au moindre claquement de porte.\\

J'ai laissé le cahier sur la table durant deux heures, et comme il bougeait pas et qu'il n'a pas explosé je m'en suis finalement emparé. Faudra que je pense à appeler Gudulle pour lui dire merci. C'est ce qui se fait dans ce genre de situation. Les gens laissent un cadeau et ceux qui le reçoivent formulent des gentillesse en retour, c'est comme ça.\\

Au fait, pendant que j'y pense, je ne me suis même pas présenté. On m'appelle, enfin les rares personnes qui cherchent à attirer mon attention m'appellent Neon. Mon vrai nom c'est Nicodème, mais si on abrège en Nico tout le monde pense que c'est le diminutif de Nicolas, alors qu'en fait ça n'a rien à voir. Ce serait trop épuisant de corriger à tour de bras les erreurs sur ma patronymie, alors va pour Neon. \\
J'ai fêté mon trentième anniversaire et j'ai pas encore soufflé mes quarante bougies. Homo sapiens sapiens par fatalisme, pur produit de la génération X, je suis fumeur, cynique, désabusé, célibataire et pas des plus sociable. J'ai facilement une bonne douzaine de tares et défauts confondus sur le râble mais dans l'ensemble je m'en porte pas trop mal. \\

Et sinon vous ça va ? \\

En fait un cahier c'est mieux qu'une tablette, j'ai fait tombé mon cahier de la table en ramassant les miettes et il est même pas cassé. C'est cool en fait.