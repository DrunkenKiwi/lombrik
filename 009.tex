\chapter*{Jour 9}
Aujourd'hui je vais chercher mon lapin. \\
Je suis presque impatient, c'est étrange comme impression. C'est un peu comme si je m'étais enfilé un café un peu trop chargé, c'est comme des picotements dans le cerveau. \\
J'ai l'étrange sensation de franchir une étape importante sur le chemin qui me mène ... je sais pas trop où d'ailleurs, mais je suis sur la bonne route c'est sur. Faut que j’arrête de réfléchir un peu, j'ai l'impression de parler comme David Caradine dans Kung Fu. J'ai du m'endormir devant un film de karaté hier soir, ça doit être ça. La télé c'est pas très bon pour les neurones, je m'étonne que mon psy ne m'aie pas mit en garde sur ce danger. \\

Non, c'est logique, mes psychoses : c'est son gagne pain. C'est pas demain qu'il aura une conscience lui, un comble pour un freudien.\\
Je suis sur qu'en cherchant bien Sigmund aurait trouvé un symbole maternel dans le lapin. D'habitude Sigmund voit des phallus dans tout ce qui est plus long que large, mais avec le lapin c'est un peu plus compliqué. Certes il reste l'analogie sexuelle avec cette photocopieuse biologique version miniature mais c'est tellement évident qu'il ne tomberait pas dans un piège aussi grossier. \\
Si les psychologues ne se donnaient pas la peine de couper les cheveux en quatre pour aller ils laisseraient la porte ouverte ou tout à chacun pour prendre leur place. Le premier, et sans doute même le dernier, des crétins trouverait une bonne opportunité professionnelle à se proclamer maître en sciences mentales et oniriques sans autre qualification qu'un aplomb sans borne. \\
Peut-être que c'est déjà le cas. \\

Le futur ex propriétaire de mon futur lapin habite en banlieue. Attention pas la banlieue qui crains, pas la banlieue des immeubles qui poussent jusqu'à étouffer le soleil. Pas cette jungle urbaine qui où les tags courent le long des murs comme du lierre et les habitant sont retournés à l'état sauvage. L'autre banlieue de l'autre coté du périphérique, celle qui sent les vaches, la cambrousse et les prairies en fleurs. Manquerait plus qu'une petite Ingalls pour dévaler en courant la cote et le tableau serait complet. Vous vous souvenez de Melissa Sue Anderson ? Non ? Vous perdez pas grand chose en fait. \\

C'est marrant comme la nature reprend rapidement ses droits dès qu'on lui fout un peu la paix. Je suis sur que si j'avais des plantes vertes chez moi, elles profiteraient de la plus petite de mes absences pour tout coloniser et étendre une jungle de ficus sauvage du salon jusqu'à la salle de bain. Le ficus est d'un naturel expansif et colonialiste, et ce n'est pas son plus petit défaut. \\

Ça fait bien vingt minutes que je suis planté devant la porte d'entrée et visiblement personne ne semble décidé à m'ouvrir. Peut-être que ma tête ne leur revient pas, et ça pourrait se comprendre ou peut-être qu'il faudrait que je me résolve à utiliser la sonnette.\\
Quand je pense au nombre de démarcheurs, témoins de Krishna, enquêteurs et quémandeurs qui ont pressés ce petit bouton blanc de leur index, j'en viens presque à regretter de ne pas avoir pris mes gants en latex. \\

Je me lance.\\

J'entends un bruit de carillon électronique suivi de prêt par des bruits de pas, c'est bon signe. C'est toujours mieux qu'entendre les aboiements d'un molosse affamé. \\

La porte s'ouvre et j'abaisse les yeux sur un petit gamin qui dépasse tout juste de l’entrebâillement de la porte. Le gamin fait tout au plus la moitié de ma taille, il a un bonnet gris enfoncé sur le crane, un visage blanc comme de la craie ou se creusent deux grands yeux caves pointés vers moi. On dirait une version du clown blanc en plus triste et moins drôle. \\
J'ai jamais aimé les clowns, et visiblement je ne vais pas commencer aujourd'hui. \\

Usant abusivement de mon droit d’aînesse, je brise le pesant silence qui s'est abattu sur nous. \\

\begin{quote}
- Bonjour, je viens pour le lapin. Sont là tes parents ? \\
- Ma Maman dort, alors faut pas sonner pour pas la réveiller. \\
- Ah, désolé alors. Mais pour le lapin ? \\
- Entrez. \\
\end{quote}

La porte s'est ouverte en grand, et j'ai plongé dans l'ouverture. Je suis resté quelques instants à faire du sur place sur le paillasson, pour m'essuyer les pieds d'une part, et aussi pour en profiter pour scruter l'intérieur des lieux. Une couloir aux murs fraîchement peint couleur pastel donnant sur la pièce de vie dont je ne voit pas grand chose, juste devant moi un meuble avec un grand bocal de clefs et de bricoles diverses posé sur le dessus et trois paires de chaussures en dessous. \\
Maman ours, papa ours et petit ours, le compte y est comme dans le conte. \\

Petit ours s'est esquivé quelques instants, il m'a demandé de l'attendre ici. \\
Ça m'arrange, je ne me sens pas de faire le tour du propriétaire. \\
C'est trop figé ici, je me sens pas trop à l'aise, pas du tout même. Je commence à sentir monter en moi les vagues caractéristiques de la crise d'angoisse. J'ai comme un étaux qui me compresse la poitrine et m'empêche de respirer normalement. Il doit y avoir quelque chose de particulièrement anxiogène dans ces murs mais je ne saurais dire quoi exactement. Je lutte de plus en plus contre l'idée irrationnelle de prendre mes jambes à mon cou et de m'enfuir en courant de ce lieu. \\
Le temps commence à devenir de plus en plus gluant, jusqu'à se figer complètement. Je sens mon cœur taper de plus en plus fort sur sa prison de cotes. Des points blancs commencent à faire leur apparition dans mon champs de vision.\\

Je sens qu'une nouvelle crise est sur le point d'éclater. \\

Petit ours est de retour, tenant dans ses bras un gigantesque rongeur gris aux oreilles tombantes. \\

\begin{quote}
- Ça va pas monsieur, t'es tout pale ? \\
- Pas vraiment, enfin je crois. Ce serait possible de m’asseoir et de prendre un verre ? Un truc vraiment fort, si c'est aussi possible. \\
- J'ai de la menthe glaciale, c'est super fort. \\
\end{quote}

J'aurais bien protesté que ce n'était pas exactement le genre de réconfort qu'il m'aurait fallu, mais je n'en avais pas la force. J'ai suivi docilement le petit ours et son lapin dans la cuisine. La cuisine était, on aurait pu s'y attendre, le modèle typique de cuisine américaine que les publicistes mettent en scène pour leur réclame. Papa Ours et Maman Ours n'ont pas du aller bien loin dans le catalogue pour arrêter leur choix. Je me suis affalé sur un tabouret de bar en prenant garde à ne pas me retrouver misérablement face contre le carrelage en attendant la suite. \\
Petit ours s'est hissé sur la pointe des pieds pour m'attraper un verre et s'est empressé de verser une solide dose de sirop vert foncé. Un rapide coup d’œil dans ma direction, et une nouvelle dose atterrie dans le verre. \\
Je dois faire peur à voir. \\

J’avale le verre d'un trait, c'est un peu comme si j'ingurgitais un grand verre de Get 27 mais sans l'alcool pour atténuer le coté écoeurant du sucre. Heureusement que je ne suis pas diabétique sinon je serais tombé raide mort d'une crise d'hyperglycémie foudroyante. Je crois que j'ai fait le plein de glucose pour au moins un bon mois. \\
Étrangement ça me requinque un peu.\\
Petit ours brun me regarde toujours de ses yeux creux, la bouteille dans une main, grandes oreilles dans l'autre. \\
Je tenterais bien un sourire de mes lèvres collantes pour le rassurer, mais je n'arriverais tout au mieux qu'a faire une grimace ridicule. A chaque fois que je me risque à retrousser les commissures de mes lèvres je ne produis qu'un regard effrayé de la part de mon interlocuteur. \\
Petit ours n'a sans doute pas besoin de ça. \\

\begin{quote}
Ça va mieux monsieur ? T'en veux un autre ?
\end{quote}

Je parviens tant que mal à articuler un oui. Le sirop ma quasiment collé les lèvres et j'ai du mal à ouvrir la bouche. \\

Le lapin, toujours impassible, me regarde de ses grands yeux vides de rongeur. Il, le lapin s'entend, semble mâchonner un chewing-gum invisible et ne se préoccuper de rien d'autre. \\

Mes yeux se portent à nouveau sur petit ours, il ressemble plus à une poupée cassée qu'à un clown finalement. Ses pommettes sont saillantes et ses yeux sont tellement cernés de noir qu'on les dirait tombés au fond d'un puit. \\

\begin{quote}
- Il te plaît mon lapin, monsieur ? Il s'appelle Findus et c'est mon copain. \\
- Ah ... \\
- Il est très gentil, tu verras. Il saute pas partout et il grignote pas les trucs qui traînent. \\
- Je sais que c'est peut-être pas mes affaires mais si c'est ton copain pourquoi tu veux t'en débarrasser ? \\
\end{quote}

Petit ours reste un moment silencieux. \\
Je déteste ce genre de moment, je suis sur que je viens de faire un impair mais je ne saurais dire lequel. Je ne sais pas ce que font les gens dans ce genre de situation, il doit bien exister une phrase passe partout à glisser pour passer à autre chose.\\
Je serais bien heureux de savoir laquelle. \\

\begin{quote}
- Maman dit que je dois bientôt rentrer à l'hopital et ils pourront pas me garder Findus. Alors il faut que je le laisse à quelqu'un pour qu'il s'en occupe bien. Comme ça quand j'irais mieux je pourrais le récupérer. \\
Mais tu sais monsieur, je crois pas que je vais aller mieux, mais il ne faut pas le dire à Maman. Tu promets, monsieur.
\end{quote}

Il y a des jours, ou je me dis que l'existence est une sacré pute.