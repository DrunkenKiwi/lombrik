\chapter*{Jour 11}
La mort de Petit Ours m'a fait tout drôle. \\
Je le connaissais pas ce gamin, je ne l'ai vu qu'une fois en tout et pour tout. Je connaissais même pas son prénom, ni rien du tout sur lui en fait. \\
C'est La Mort qui m'a annoncé sa mort et ça ma fait tout drôle. \\

Faudrait que je l'encadre cette phrase. Rien que de la relire ça me chiffonne. \\

Je me sens encore plus abattu que d'habitude. Même Findus est triste, je peux lire la tristesse sur sa face de lapin. Ses oreilles traînent plus bas que terre, et son regard est comme brouillé par un voile de larmes. Il ne semble ne plus avoir goût à sa nourriture et passe le plus clair de son temps prostré derrière une vieille toile inachevée. \\
Je ne sais pas si d'habitude les lapins pleurent leur maître mais c'est le cas du mien. \\

Faudrait que je trouve quelque chose pour lui remonter le moral avant qu'il ne devienne trop contagieux. Il ne manque pas grand chose pour pousser ma petite vieille de voisine au suicide. \\
Faut dire que je l'ai pas vraiment aidé avec son chat, même si techniquement c'est Snow qui est la coupable. \\
Pour être franc je ne pense pas que se serait une grosse perte, enfin en ce qui me concerne. D'un autre coté je ne ferais rien pour précipiter les choses car je ne suis certainement pas mentionné dans son testament. \\

Je ne vais pas emmener Findus voir mon psy. \\
Tout d'abord parce que c'est le mien, c'est moi qui l'ai vu en premier et d'autre part car je doute vraiment qu'il puisse faire quoi que ce soit pour lui. Est ce qu'il m'a aidé moi ? Non, alors. Et puis de toute façon je n'aime pas les réducteurs de tête. Et c'est un avis définitif. \\

Je ne vais pas non plus emmener Findus au parc d'attractions. \\
Déjà je ne sais pas s'ils acceptent les animaux et quand bien même se serait le cas, je n'irais pas. C'est plein de clowns et de gens bizarres qui se déguisent en personnage de dessin animé. Ils me font peur avec leurs costumes grotesques et leurs poses outrancières. Je ne supporte pas plus leurs chorégraphies pitoyables et anxiogènes. \\
Et puis on y ferait quoi là bas ? Je te le demande Findus. \\
Tu t'imagines sur le grand huit avec les oreilles claquant dans le vents et les babines troussées par la vitesse. ? \\
Non ? Et bien moi non plus. \\

On va au parc alors ? \\

Tu as raison Findus. \\
C'est toujours ce que je fais quand j'ai pas le moral, je vais au parc. Je te dis ça mais remarques moi j'ai jamais perdu mon maître. En fait, j'ai pas de maître mais si j'en avais eu un je suis sur que je serais allé au parc.\\
Me regardes pas comme ça Findus, j'essaie de t'aider du mieux que je peux. \\

J'ai rattrapé mon manteau, j'ai fourré Findus dans mon sac et j'ai pris quelques carottes que j'avais spécialement acheté pour lui dans mes poches. \\

Ça me gène de me proclamer le maître de Findus. Je ne me sens pas plus son maître que son propriétaire, ça n'a aucun sens de dire : c'est mon lapin. Il ne va pas accourir si je le siffle (enfin je ne pense pas), il ne m'obéira pas si lui donne un ordre. C'est pas un chien et c'est pas non plus dans ma nature d'aboyer. \\
Je n'ai pas de titre en bonne et due forme attestant de mon emprise sur lui. J'ai même pas de reçu, de facture, ni de talon de chèque. \\
Findus c'est un peu comme un colocataire qui ne paierais pas sa part de loyer mais qui n'inviterait pas de copains à s'arsouiller dans mon salon ou qui chercherait à faire rentrer discrètement ses conquêtes d'un soir.

Je suis sorti, Findus sur mon dos. Je sentais son haleine chaude dans le cou et le poids de son regard sur mon épaule. \\
Je suis sorti, Findus sur mon dos et nous sommes allé au parc. \\

Le parc c'est bien, c'est pas loin et en plus il y a des canards. Je me suis souvent demandé si les canards et les lapins faisaient aussi bon ménage que dans les gravures de Maurits Cornelis Escher. En fait si vous voulez tout savoir ils s'ignorent royalement et ça me convient très bien car je suis rarement d'humeur à arbitrer des duels et aujourd'hui encore moins que d'habitude. \\
Ce parc c'est un peu mon havre de paix. L’absence de terrain de jeux, fait que ce lieu est relativement délaissé par les enfants. Exit les hordes de braillards baveux et sautillants. Les chiens aussi semblent éviter le lieu, désolé si la transition vous parait brutale mais j'assume pleinement le parallèle entre les deux sentences précédentes. Les chiens sont autorisés, ils leurs ai juste interdit de divaguer. Ce n'est pas moi qui l'invente c'est inscrit en toutes lettres sur un panneau prêt de l'entrée. \\

Je me suis assis dans l'herbe, j'ai hésité un bon moment avant de sortir Findus de mon sac. S'il avait décidé de prendre la poudre d'escampette j'aurais bien été en peine de le rattraper et puis j'aurais eu l'air de quoi vis à vis de La Mort. \\
J'ai regardé Findus dans le marron des yeux et je lui ai fait promettre de ne pas s'enfuir. \\
Il n'a rien dit mais j'ai pris ça pour un acquiescement.\\

Nous sommes resté un bon moment tout les deux à regarder les canards s'ébrouer dans la mare. Ils sont très forts pour ça les canards, pour le reste : ça reste des canards alors il faut pas trop leurs en demander.\\

Il faudra que je pense amener Findus à l'enterrement de Petit Ours. \\
Ça m'enchante pas plus que ça mais il faut bien que je le fasse. Gaïl ne pense pas que ce soit une bonne idée mais elle ne pense pas non plus que je ne devrais pas y aller. Il faut juste que je fasse attention, si je dis aux parents du petit défunt que c'est la mort qui m'a convié à la cérémonie funèbre elle pense qu'ils le prendraient mal. \\
Gaïl m'a fortement conseillé de m'abstenir de parler d'elle, surtout à mon psy. Elle pense qu'il ne comprendrait pas et que cela pourrait aggraver mon cas. \\

Elle à raison Gaïl, elle a toujours raison.

